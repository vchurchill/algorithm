\documentclass[12pt, letterpaper]{article}
\usepackage[utf8x]{inputenc}
\usepackage{fullpage}
\usepackage{amsmath}
\usepackage{amsfonts}
\usepackage{graphicx}
\usepackage{subfig}
\usepackage[pdftex,bookmarks=true]{hyperref}
\usepackage{array}
\usepackage[parfill]{parskip}
\usepackage{multirow}
\usepackage{framed}
\usepackage{csquotes}
\usepackage{alltt}
\usepackage{cancel}
\usepackage{breakurl}
\usepackage{wrapfig}
\usepackage{comment}
\usepackage{units}
\usepackage{fancybox}
\usepackage{amssymb}
\usepackage{tikz}
\usepackage{pgfplots}
\usepgfplotslibrary{fillbetween}
\usetikzlibrary{arrows}
\usetikzlibrary{pgfplots.groupplots}
\usetikzlibrary{matrix,positioning}
\usepackage{tkz-euclide}
\usetkzobj{all}

\captionsetup[subfigure]{labelformat=empty, labelsep=none}

\newcommand{\HRule}{\rule{\linewidth}{0.2mm}}
\begin{titlepage}
\title{\vspace{-3ex}
{\bf \Large PDE I Final HW} \vspace{-1ex}}
\author{Victor Churchill}
\date{}
\end{titlepage}
\begin{document}
\maketitle

I chose the second problem.

$\mathbf{II}$
\begin{enumerate}
\item 

To obtain an equation with $||u||^2_{L^2}$, we multiply $(10)$ by $u$ and strategically simplify to get

\begin{align}
\big(\frac{1}{2}u^2\big)_t+|\nabla u|^2-\nabla \cdot(u\nabla u)+uf(u)=0
\end{align}

Integrating over $\Omega$ and using the divergence theorem and our boundary condition yields:

\begin{align}
\frac{1}{2}\frac{d}{dt}||u||_{L^2}^2+||\nabla u||^2_{L^2}+\int_\Omega u f(u)dx=0
\end{align}

This is our equation with $||u||^2_{L^2}$.

Now we must show $E(u)$ is decreasing. To show $E(u)$ is decreasing, we need to show $\partial_t\big[E(u)\big]\le 0$. Start with:

\begin{align}
E(u)&=\frac{1}{2}\int_\Omega\nabla u \nabla u dx+ \int_\Omega F(u)dx\\
&=-\frac{1}{2}\int_\Omega u\Delta u dx + \int_\Omega F(u)dx
\end{align}

Taking the derivative of that equation with respect to time, we have:

\begin{align}
\frac{d}{dt}E(u)&=-\frac{1}{2}\int_\Omega [u_t\Delta u + u\Delta u_t]dx +\int_\Omega u_tF'(u)dx\\
&=\int_\Omega \big[-u_t\Delta u+ u_tf(u)\big]dx
\end{align}

Now we need to use the given info. Multiplying $(10)$ by $u_t$ and integrating over $\Omega$, we have:

\begin{align}
\int_\Omega\big[u_t^2-u_t\Delta u +u_tf(u)\big]dx=0
\end{align}

This means that:

\begin{align}
\frac{d}{dt}E(u)=-\int_\Omega u_t^2dx \implies \frac{d}{dt}E(u)\le 0
\end{align}

as desired. Therefore $E(u)$ is decreasing.

\item We assume $g\in L^2(\Omega)$. Since $|f(u)|\le C(1+|u|)$, then $|uf(u)|$ is less than or equal to a polynomial of degree $2$ with positive leading coefficient. So there exist constants $\alpha>0$, $\beta\ge0$ such that

\begin{align}
\alpha||u||^2_{L^2}\le\int_\Omega uf(u)dx + \beta
\end{align}

Therefore we have

\begin{align}
\frac{1}{2}\sup_{[0,T]}||u||^2_{L^2} + \int_0^T||\nabla u||^2_{L^2}dt +\alpha\int_0^T||u||_{L^2}^2dt \le\beta T+\frac{1}{2}||g||^2_{L^2}
\end{align}

Note if $||u||_{L^2}\le\infty$ then $||f(u)||_{L^2}\le\infty$ since

\begin{align}
\int_\Omega |f(u)|^2dx\le \alpha\int_\Omega|u|^2dx+\beta\le\alpha||u||_{L^2}+\beta
\end{align}

So in forming a weak solution we use test functions: $v\in H_0^1(\Omega)\cap L^2(\Omega)$ such that $(\nabla u, \nabla v)_{L^2}$ and $(f(u),v)_{L^2}$ are well-defined.

The Galerkin approximations $\{u_m\}$ take values in a finite dimensional subspace $E_m\subset H_0^1(\Omega)\cap L^2(\Omega)$ and satisfy the following with $P_m$ the projection of $E_m$ onto $L^2(\Omega)$

\begin{align}
(u_m)_t=\Delta u_m + P_m f(u_m)
\end{align}

Similar to the solution $u$, these approximations satisfy

\begin{align}
\frac{1}{2}\sup_{[0,T]}||u_m||^2_{L^2} + \int_0^T||\nabla u_m||^2_{L^2}dt +\alpha\int_0^T||u_m||_{L^2}^2dt \le\beta T+\frac{1}{2}||g||^2_{L^2}
\end{align}

The Galerkin ODEs have a unique local solution since the nonlinear terms are Lipschitz continuous functions of $u_m$. Furthermore, the local solutions are bounded, and so they exist globally for all times $t\ge0$.

Since our estimates hold uniformly for $m$, we can pick a subsequence that weakly converges $u_m\rightarrow u$ to a limiting function

\begin{align}
u\in L^\infty(0,T;L^2(\Omega))\cap L^2(0,T;H^1_0)\cap L^2(0,T;L^2)
\end{align}

And furthermore we can say that $u_t\in L^2(0,T;H^{-1})+ L^2(0,T;L^2)$.

Now we need to show that this $u$ is actually a solution of the original PDE. To do this, we must show $f(u_m)\rightarrow f(u)$. We find some subsequence of $u_m$ such that $u_m\rightarrow u$ strongly in $L^2(0,T;L^2)$, which is the same as a strong $L^2$ convergence for $\Omega \times (0,T)$. By the Riesz-Fisher theorem, we can then extract a subsequence such that $u_m(x,t)\rightarrow u(x,t)$ pointwise on $\Omega \times (0,T)$. Then we use the dominated convergence theorem and the uniform bounds of $u_m$ to find that for every test function $v\in H_0^1(\Omega)\cap L^2(\Omega)$

\begin{align}
(f(u_m(t)),v)_{L^2}\rightarrow(f(u(t)),v)_{L^2}
\end{align}

pointwise on $[0,T]$.

So here we've shown that the PDE has a global unique weak solution with the desired properties.


\item \begin{enumerate}
\item Now we have $g\in H_0^1(\Omega)$ and $uf(u)$ is a monomial of degree $p+1$ with $1<p<5$ and leading coefficient $\lambda$.

There exist constants $\alpha>0$, $\beta\ge0$ such that

\begin{align}
\alpha||u||^2_{L^{p+1}}\le\int_\Omega uf(u)dx + \beta
\end{align}

Therefore we have

\begin{align}
\frac{1}{2}\sup_{[0,T^*)}||u||^2_{L^{2}} + \int_0^{T^*}||\nabla u||^2_{L^{2}}dt +\alpha\int_0^{T^*}||u||_{L^{p+1}}^2dt \le\beta T^*+\frac{1}{2}||g||^2_{H_0^1}
\end{align}

Note if $||u||_{L^{p+1}}\le\infty$ then $||f(u)||_{L^{\frac{p+1}{p}}}\le\infty$ since

\begin{align}
\int_\Omega |f(u)|^{\frac{p+1}{p}}dx\le \alpha\int_\Omega|u|^{p+1}dx+\beta\le\alpha||u||_{L^{p+1}}+\beta
\end{align}

Note that we use $\frac{p+1}{p}$ in this case since it is the Holder conjugate of $p+1$. So in forming a weak solution we use test functions: $v\in H_0^1(\Omega)\cap L^{p+1}(\Omega)$ such that $(\nabla u, \nabla v)_{L^2}$ and $(f(u),v)_{L^2}$ are well-defined.

The Galerkin approximations $\{u_m\}$ take values in a finite dimensional subspace $E_m\subset H_0^1(\Omega)\cap L^{p+1}(\Omega)$ and satisfy the following with $P_m$ the projection of $E_m$ onto $L^2(\Omega)$

\begin{align}
(u_m)_t=\Delta u_m + P_m f(u_m)
\end{align}

Similar to the solution $u$, these approximations satisfy

\begin{align}
\frac{1}{2}\sup_{[0,T^*)}||u_m||^2_{L^2} + \int_0^{T^*}||\nabla u_m||^2_{L^2}dt +\alpha\int_0^{T^*}||u_m||_{L^{p+1}}^2dt \le\beta T^*+\frac{1}{2}||g||^2_{H_0^1}
\end{align}

The Galerkin ODEs have a unique local solution since the nonlinear terms are Lipschitz continuous functions of $u_m$. Furthermore, the local solutions are bounded, and so they exist globally for all times $t\ge0$.

Since our estimates hold uniformly for $m$, we can pick a subsequence that weakly converges $u_m\rightarrow u$ to a limiting function

\begin{align}
u\in L^\infty([0,T^*);H_0^1\cap L^{p+1}(\Omega))\cap L^2([0,T^*);H^1_0)\cap L^{p+1}([0,T^*)\times\Omega)
\end{align}

Now we need to show that this $u$ is actually a solution of the original PDE. To do this, we must show $f(u_m)\rightarrow f(u)$. We find some subsequence of $u_m$ such that $u_m\rightarrow u$ strongly in $L^{p+1}([0,T^*)\times\Omega)$. By the Riesz-Fisher theorem, we can then extract a subsequence such that $u_m(x,t)\rightarrow u(x,t)$ pointwise on $[0,T^*)\times\Omega$. Then we use the dominated convergence theorem and the uniform bounds of $u_m$ to find that for every test function $v\in H_0^1(\Omega)\cap L^{p+1}(\Omega)$

\begin{align}
(f(u_m(t)),v)_{L^2}\rightarrow(f(u(t)),v)_{L^2}
\end{align}

pointwise on $[0,T^*)$.

So here we've shown that the PDE has a global unique weak solution with the desired properties. In particular, $T^* >0$ exists.

\item So part $(a)$ shows that there exists a unique solution defined on some maximal time interval $[0,T^*)$, $u\in L^\infty([0,T^*);H_0^1\cap L^{p+1}(\Omega))$ for all $T<T^*$. For $\lambda\ge0$,this means there is a potential blowup: either $T^*=+\infty$ or $T^*<\infty$ and $||u||_{L^\infty}\rightarrow\infty$ as $t\rightarrow T^*$. However $||u||_{L^\infty}$ is bounded, so $T^*=+\infty$.

\item \begin{enumerate}
\item First, $-4E(g)=-2||\nabla g||^2_{L^2(\Omega)}-4\int_\Omega F(g)dx > 0$. Also $y^{\frac{p+1}{2}}=||u||^{p+1}_{L^2(\Omega)}$. Lastly, $\dot y=2u_t||u||_{L^2(\Omega)}$.

We also need to use Holder's inequality $||f||_p||g||_q\ge||fg||_1$ is $\frac{1}{p}+\frac{1}{q}=1$, and also $\int_\Omega |f(x)g(x)|dx\le (\int_\Omega|f(x)|^pdx)^{\frac{1}{p}}(\int_\Omega|g(x)|^qdx)^{\frac{1}{q}}$.

Maybe an exponential function?

Unfortunately not sure where to take it from here.

\item For this to be true we would need $2\int_\Omega F(g)dx < -||\nabla g||^2_{L^2(\Omega)}$.

\end{enumerate}

\end{enumerate}


\item \begin{enumerate}
\item We have $g\in L^2(\Omega)$ and $uf(u)$ is some $p+1$ degree monomial with $1<p<5$ and positive leading coefficient $\lambda$.

There exist constants $\alpha>0$, $\beta\ge0$ such that

\begin{align}
\alpha||u||^2_{L^{p+1}}\le\int_\Omega uf(u)dx + \beta
\end{align}

Therefore we have

\begin{align}
\frac{1}{2}\sup_{[0,T]}||u||^2_{L^{2}} + \int_0^T||\nabla u||^2_{L^{2}}dt +\alpha\int_0^T||u||_{L^{p+1}}^2dt \le\beta T+\frac{1}{2}||g||^2_{L^2}
\end{align}

Note if $||u||_{L^{p+1}}\le\infty$ then $||f(u)||_{L^{\frac{p+1}{p}}}\le\infty$ since

\begin{align}
\int_\Omega |f(u)|^{\frac{p+1}{p}}dx\le \alpha\int_\Omega|u|^{p+1}dx+\beta\le\alpha||u||_{L^{p+1}}+\beta
\end{align}

Note that we use $\frac{p+1}{p}$ in this case since it is the Holder conjugate of $p+1$. So in forming a weak solution we use test functions: $v\in H_0^1(\Omega)\cap L^{p+1}(\Omega)$ such that $(\nabla u, \nabla v)_{L^2}$ and $(f(u),v)_{L^2}$ are well-defined.

The Galerkin approximations $\{u_m\}$ take values in a finite dimensional subspace $E_m\subset H_0^1(\Omega)\cap L^{p+1}(\Omega)$ and satisfy the following with $P_m$ the projection of $E_m$ onto $L^2(\Omega)$

\begin{align}
(u_m)_t=\Delta u_m + P_m f(u_m)
\end{align}

Similar to the solution $u$, these approximations satisfy

\begin{align}
\frac{1}{2}\sup_{[0,T]}||u_m||^2_{L^2} + \int_0^T||\nabla u_m||^2_{L^2}dt +\alpha\int_0^T||u_m||_{L^{p+1}}^2dt \le\beta T+\frac{1}{2}||g||^2_{L^2}
\end{align}

The Galerkin ODEs have a unique local solution since the nonlinear terms are Lipschitz continuous functions of $u_m$. Furthermore, the local solutions are bounded, and so they exist globally for all times $t\ge0$.

Since our estimates hold uniformly for $m$, we can pick a subsequence that weakly converges $u_m\rightarrow u$ to a limiting function

\begin{align}
u\in L^\infty(0,T;L^2(\Omega))\cap L^2(0,T;H^1_0)\cap L^{p+1}((0,T)\times\Omega)
\end{align}

And furthermore we can say that $u_t\in L^2(0,T;H^{-1})+ L^\frac{p+1}{p}((0,T)\times\Omega)$.

Now we need to show that this $u$ is actually a solution of the original PDE. To do this, we must show $f(u_m)\rightarrow f(u)$. We find some subsequence of $u_m$ such that $u_m\rightarrow u$ strongly in $L^{p+1}((0,T)\times\Omega)$. By the Riesz-Fisher theorem, we can then extract a subsequence such that $u_m(x,t)\rightarrow u(x,t)$ pointwise on $(0,T)\times\Omega$. Then we use the dominated convergence theorem and the uniform bounds of $u_m$ to find that for every test function $v\in H_0^1(\Omega)\cap L^{p+1}(\Omega)$

\begin{align}
(f(u_m(t)),v)_{L^2}\rightarrow(f(u(t)),v)_{L^2}
\end{align}

pointwise on $[0,T]$.

So here we've shown that the PDE has a global unique weak solution with the desired properties.

\item Following the same proof above for a $g\in H_0^1\cap L^{p+1}$ and for every $T>0$ we will get a unique solution $u\in L^\infty(0,T;H_0^1\cap L^{p+1}(\Omega))\cap L^2(0,T;H^1_0)\cap L^{p+1}((0,T)\times\Omega)$ which of course means $u\in L^\infty(0,T;H_0^1\cap L^{p+1}(\Omega))$. Perhaps I missed the point of this question and there is some property about the $H_0^1$ space that makes the conclusion clear from the above result.

\item For both $(a)$ and $b$, I believe yes uniqueness is possible, as in my proofs.

\end{enumerate}

\end{enumerate}

Thanks for a great semester! Happy New Year!

\end{document}