%% NYU PhD thesis format. Created by Jos� Koiller 2007--2008.

%% Use the first of the following lines during production to
%% easily spot "overfull boxes" in the output. Use the second
%% line for the final version.
%\documentclass[12pt,draft,letterpaper]{report}
\documentclass[12pt,letterpaper]{report}

%% Replace the title, name, advisor name, graduation date and dedication below with
%% your own. Graduation months must be January, May or September.
\newcommand{\thesistitle}{Proof of the Riemann Hypothesis}
\newcommand{\thesisauthor}{Jane Doe}
\newcommand{\thesisadvisor}{Professor Fulana de Tal}
\newcommand{\graddate}{May 2032}
%% If you do not want a dedication, scroll down and comment out
%% the appropriate lines in this file.
\newcommand{\thesisdedication}{To my dog Weierstra\ss, with affection.}

%% The following makes chapters and sections, but not subsections,
%% appear in the TOC (table of contents). Increase to 2 or 3 to
%% make subsections or subsubsections appear, respectively. It seems
%% to be usual to use the "1" setting, however.
\setcounter{tocdepth}{1}

%% Sectional units up to subsubsections are numbered. To number
%% subsections, but not subsubsections, decrease this counter to 2.
\setcounter{secnumdepth}{3}

%% Page layout (customized to letter paper and NYU requirements):
\setlength{\oddsidemargin}{.6in}
\setlength{\textwidth}{5.8in}
\setlength{\topmargin}{.1in}
\setlength{\headheight}{0in}
\setlength{\headsep}{0in}
\setlength{\textheight}{8.3in}
\setlength{\footskip}{.5in}

%% Use the following commands, if desired, during production.
%% Comment them out for final version.
%\usepackage{layout} % defines the \layout command, see below
%\setlength{\hoffset}{-.75in} % creates a large right margin for notes and \showlabels

%% Controls spacing between lines (\doublespacing, \onehalfspacing, etc.):
\usepackage{setspace}

%% Use the line below for official NYU version, which requires
%% double line spacing. For all other uses, this is unnecessary,
%% so the line can be commented out.
\doublespacing % requires package setspace, invoked above

%% Each of the following lines defines the \com command, which produces
%% a comment (notes for yourself, for instance) in the output file.
%% Example:    \com{this will appear as a comment in the output}
%% Choose (uncomment) only one of the three forms:
%\newcommand{\com}[1]{[/// {#1} ///]}       % between [/// and ///].
\newcommand{\com}[1]{\marginpar{\tiny #1}} % as (tiny) margin notes
%\newcommand{\com}[1]{}                     % suppress all comments.

%% This inputs your auxiliary file with \usepackage's and \newcommand's:
%% It is assumed that that file is called "definitions.tex".
\input{definitions}

%% Cross-referencing utilities. Use one or the other--whichever you prefer--
%% but comment out both lines for final version.
%\usepackage{showlabels}
%\usepackage{showkeys}


\begin{document}
%% Produces a test "layout" page, for "debugging" purposes only.
%% Comment out for final version.
%\layout % requires package layout (see above, on this same file)

%%%%%% Title page %%%%%%%%%%%
%% Sets page numbering to "roman style" i, ii, iii, iv, etc:
\pagenumbering{roman}
%
%% No numbering in the title page:

%

%%%%%%%%%%%%% Blank page %%%%%%%%%%%%%%%%%%



%%%%%%%%%%%%%% Dedication %%%%%%%%%%%%%%%%%
%% Comment out the following lines if you do not want to dedicate
%% this to anyone...

%%%%%%%%%%%%%% Acknowledgements %%%%%%%%%%%%
%% Comment out the following lines if you do not want to acknowledge
%% anyone's help...

%%%% Abstract %%%%%%%%%%%%%%%%%%

%%%% Table of Contents %%%%%%%%%%%%

%%%%% List of Figures %%%%%%%%%%%%%
%% Comment out the following two lines if your thesis does not
%% contain any figures. The list of figures contains only
%% those figures included withing the "figure" environment.


%%%%% List of Tables %%%%%%%%%%%%%
%% Comment out the following two lines if your thesis does not
%% contain any tables. The list of tables contains only
%% those tables included withing the "table" environment.

%%%%% Body of thesis starts %%%%%%%%%%%%
\pagenumbering{arabic} % switches page numbering to arabic: 1, 2, 3, etc.
%% Introduction. If your thesis has no introduction, or chapter 1 is
%% meant to be the introduction, then comment out the lines below.

%% If your thesis has different "Parts", use commands such as the following:
%\part{First Part\label{part:one}}%
\section{Statement of problem\label{chap:one}}

Suppose we are solving the Dirichlet problem for the Laplace equation in three dimensions on a surface of revolution such that:
\begin{align}
\Delta{u} &= 0 & \mbox{in }\Omega\\
u &= f & \mbox{on }\Gamma\\
u &\rightarrow 0 & \mbox{as } \mathbf{x}\rightarrow\infty
\end{align}

Recall that the kernel-independent fast multipole method uses a continuous distribution of an equivalent density on a surface enclosing a box in the quad- or octree to represent the potential generated by sources in that box, rather than using analytic multipole expansions like the original FMM. This allowed us to construct an FMM that only requires kernel evaluations without sacrificing efficiency. It is also relatively easier to implement, since it requires very few changes to apply to different kernels.

The original KIFMM explored three data sets for the 3D case: densities distributed on the unit sphere, densities distributed uniformly on the unit cube, and densities distributed at the eight corners of the unit cube [Ying, Biros, Zorin]. This project examines the case of another non-uniform distribution, surfaces of revolution. In particular, surfaces of revolution that are rotationally symmetric with respect to the azimuthal angle, or axisymmetric. In trying to construct a KIFMM for this case, there are several issues. I propose that we use [Yao, Martinsson, and Young]'s technique of 

The KIFMM is similar to the FMM, apart from how the equivalent densities are represented efficiently, and how the translation operators are computed.

There are two steps, a potential evaluation, and the solve of an integral equation. In particular, the evaluation of the check potential using the original sources, and the inversion of the integral equation to obtain equivalent density. Both steps require discretization. For the integral equation solve, [YBZ] uses Tikhonov regularization.

In the original KIFMM, translation operators are a pre-computation that only differ based on relative position and level in the hierarchical tree. Here, however, the kernel is not so simple. We must look for some other constant relationship between translation operators.

Requirements for KIFMM: smoothness and uniqueness, satisfied if and only if the equivalent surfaces do not intersect, etc...

There are problems that arise for this case.

Left to do: implement actual Green's function, accelerate using SVD in 2D or FFT in 3D.

\begin{center}
M2L: $\int_{\mathbf{y}^{B,d}}{G(\mathbf{x},\mathbf{y})}\phi^{B,d}{(\mathbf{y})}d\mathbf{y}$ = $\int_{\mathbf{y}^{A,u}}{G(\mathbf{x},\mathbf{y})}\phi^{A,u}{(\mathbf{y})}d\mathbf{y}$ for all $\mathbf{x}\in\mathbf{x}^{B,d}$
\end{center}

\section{The Riemann Hypothesis\label{sec:hypothesis}}

\section{Another section\label{sec:two}}


%\input{chap2} % further chapters -- change file names to meaningful things...
%\input{chap3}
%\part{Second Part\label{part:two}}%
%\input{chap4}
%\input{chap5}
%\input{chap6}
%%%%% Appendices start %%%%%%%%%%%%%%%%
%% Comment out the following line if your thesis has no appendix

%% Note: If your thesis has more than one appendix, NYU requires a "list of
%% appendices" page before the body of the thesis. I don't provide the tools
%% to create that here, so you're on your own for that one... Sorry.
%\input{app2}
%%%% Input bibliography file %%%%%%%%%%%%%%%
\begin{thebibliography}{99}\addcontentsline{toc}{chapter}{Bibliography}
\bibitem{JBC91}J.~B.~Conway, \emph{Functions of One Complex
    Variable~I}. Second edition. Springer-Verlag, Graduate
    Texts in Mathematics~\textbf{11}, 1991.
\end{thebibliography}

\end{document}
